% !TEX TS-program = xelatex
% !TEX encoding = UTF-8 Unicode
% -*- coding: UTF-8; -*-
% vim: set fenc=utf-8

%%%%%%%%%%%%%%%%%%%%%%%%%%%%%%%%%%%%%%%%%%%%%%%%%%%%%%%%%%%%%%%%%
%% SIMPLE-RESUME-CV
%% <https://github.com/zachscrivena/simple-resume-cv>
%% This is free and unencumbered software released into the
%% public domain; see <http://unlicense.org> for details.
%%%%%%%%%%%%%%%%%%%%%%%%%%%%%%%%%%%%%%%%%%%%%%%%%%%%%%%%%%%%%%%%%

%%%%%%%%%%%%%%%%%%%%%%%%%%%%%%%%%%%%%%%%%%%%%%%%%%%%%%%%%%%%%%%%%
%% INSTRUCTIONS FOR COMPILING THIS DOCUMENT ("CV.tex")
%% TeX ---(XeLaTeX)---> PDF:
%%
%% Method 1: Use latexmk for fully automated document generation:
%%   latexmk -xelatex "CV.tex"
%%   (add the -pvc switch to automatically recompile on changes)
%%
%% Method 2: Use XeLaTeX directly:
%%   xelatex "CV.tex"
%%   (run multiple times to resolve cross-references if needed)
%%%%%%%%%%%%%%%%%%%%%%%%%%%%%%%%%%%%%%%%%%%%%%%%%%%%%%%%%%%%%%%%%

\documentclass[letterpaper,MMMyyyy,nonstop]{simpleresumecv}
% Class options:
% a4paper, letterpaper, draft, nonstop
% MMMyyyy, ddMMMyyyy, MMMMyyyy, ddMMMMyyyy, yyyyMMdd, yyyyMM, yyyy

%%%%%%%%%%%%%%%%%%%%%%%%%%%%%%%%%%%%%%%%%%%%%%%%%%%%%%%%%%%%%%%%%
%% PREAMBLE.
%%%%%%%%%%%%%%%%%%%%%%%%%%%%%%%%%%%%%%%%%%%%%%%%%%%%%%%%%%%%%%%%%

% CV Info (to be customized).
\newcommand{\CVAuthor}{Jayendra Parmar}
\newcommand{\CVTitle}{John Doe's CV for Acme Corporation}
\newcommand{\CVNote}{CV compiled on {\today} for Acme Corporation}
\newcommand{\CVWebpage}{http://www.example.com/johndoe}

% PDF settings and properties.
\hypersetup{
pdftitle={\CVTitle},
pdfauthor={\CVAuthor},
pdfsubject={\CVWebpage},
pdfcreator={XeLaTeX},
pdfproducer={},
pdfkeywords={},
pdfpagemode={},
unicode=true,
bookmarks=true,
bookmarksopen=true,
pdfstartview=FitH,
pdfpagelayout=OneColumn,
pdfpagemode=UseOutlines,
hidelinks,
breaklinks}

% Shorthand.
\newcommand{\CodeCommand}[1]{\mbox{\textbf{\textbackslash{#1}}}}

%%%%%%%%%%%%%%%%%%%%%%%%%%%%%%%%%%%%%%%%%%%%%%%%%%%%%%%%%%%%%%%%%
%% ACTUAL DOCUMENT.
%%%%%%%%%%%%%%%%%%%%%%%%%%%%%%%%%%%%%%%%%%%%%%%%%%%%%%%%%%%%%%%%%

\begin{document}

%%%%%%%%%%%%%%%
% TITLE BLOCK %
%%%%%%%%%%%%%%%

\title{\CVAuthor}

\begin{subtitle}
\href{https://www.google.com/maps/place/17+Prime+Avenue,+Springfield,+Pennsylvania+10111,+USA}
{17 Prime Avenue, Apt 37, Springfield, Pennsylvania 10111, USA}
\par
\href{mailto:jayendra0parmar@gmail.com}
{jayendra0parmar@gmail.com}
\,\SubBulletSymbol\,
(+91)9978587584
\,\SubBulletSymbol\,
\href{\CVWebpage}
{\CVWebpage}
\end{subtitle}

\begin{body}

%%%%%%%%%%%%%%%
%% EDUCATION %%
%%%%%%%%%%%%%%%

\section
{Education}
{Education}
{PDF:Education}

\href{http://www.example.com/my-university}
{\textbf{First American University}},
Springfield, Massachusetts, USA

\GapNoBreak
\BulletItem
Doctor of Philosophy (Ph.D.) in
\href{http://www.example.com/my-department}
{Geophysical Engineering}
\hfill
\DatestampYMD{2009}{12}{15} --
\DatestampYMD{2014}{07}{15}
\begin{detail}
\SubBulletItem
Thesis:
\href{http://www.example.com/my-phd-thesis}
{A Statistical Approach to Quantifying Climate Change}
\SubBulletItem
Adviser:
Prof.~Jonathan~Public
\SubBulletItem
Focus:
Climate change, metrology, lasers, statistics.
\end{detail}

\GapNoBreak
\BulletItem
Master of Business Administration (M.B.A.)
\hfill
\DatestampYMD{2008}{11}{15} --
\DatestampYMD{2009}{06}{15}

\GapNoBreak
\BulletItem
Master of Science (M.S.) in
\href{http://www.example.com/my-department}
{Geophysical Engineering}
\hfill
\DatestampYMD{2006}{08}{15} --
\DatestampYMD{2008}{08}{15}
\begin{detail}
\SubBulletItem
Cumulative GPA: 3.7 / 4.0
\end{detail}

\BigGap
\href{http://www.example.com/my-college}
{\textbf{Science College}},
Springfield, Pennsylvania, USA

\GapNoBreak
\BulletItem
Bachelor of Science (B.S.) in
\href{http://www.example.com/my-department}
{Geography}
\hfill
\DatestampYMD{2002}{05}{15} --
\DatestampYMD{2005}{05}{15}
\begin{detail}
\SubBulletItem
Graduated with College Honors.
\SubBulletItem
Cumulative GPA: 3.96 / 4.00
\end{detail}

%%%%%%%%%%%%%%%%%%%%%%%%%
%% RESEARCH EXPERIENCE %%
%%%%%%%%%%%%%%%%%%%%%%%%%

\section
{Research Experience}
{Research Experience}
{PDF:ResearchExperience}

\href{http://www.example.com/my-institute}
{\textbf{Institute for Advanced Research}},
Science College

\GapNoBreak
\BulletItem
Undergraduate Research Student, Science Department
\hfill
\DatestampYMD{2004}{05}{15} --
\DatestampYMD{2005}{05}{15}
\begin{detail}
\SubBulletItem
Project:
Investigations on the Use of Lasers to Measure Climate Change
\SubBulletItem
Supervisors:
Prof.~Jane~Citizen and
Dr~Ann~Yone
\SubBulletItem
Focus:
Climate change, lasers, statistical analysis, data analytics.
\end{detail}

%%%%%%%%%%%%%%%%%%
%% PUBLICATIONS %%
%%%%%%%%%%%%%%%%%%

\section
{Publications}
{Publications}
{PDF:Publications}

\subsection
{Journals}
{Journals}
{PDF:Journals}

\GapNoBreak
\NumberedItem{[11]}
\href{http://www.example.com/my-paper-doi-5}
{\underline{J.~Doe}, J.~Citizen, and A.~Yone,
``On lasers and climate change,''
\textit{Journal of Science},
vol.~89,
no.~2,
pp.~4123--4133,
\DatestampYM{2008}{02}.}

% Note the use of {\CharSpace} for aligning shorter numbers.
\Gap
\NumberedItem{{\CharSpace}[1]}
\href{http://www.example.com/my-paper-doi-4}
{\underline{J.~Doe} and J.~Citizen,
``Measuring the extent of climate change,''
\textit{Global Scientific Journal},
vol.~12,
no.~4,
pp.~330--352,
\DatestampYM{2006}{12}.}

\BigGap
\subsection
{Conferences}
{Conferences}
{PDF:Conferences}

\GapNoBreak
\NumberedItem{[11]}
\href{http://www.example.com/my-paper-doi-3}
{\underline{J.~Doe}, J.~Citizen, and A.~Yone,
``On lasers and climate change,''
in \textit{Proceedings of the Laser Symposium},
Las Vegas, Nevada, USA,
\DatestampYM{2007}{01}.}

\Gap
\NumberedItem{[10]}
\href{http://www.example.com/my-paper-doi-2}
{A.~Yone and \underline{J.~Doe},
``Climate change and general relativity,''
in \textit{Proceedings of the International Astronomical Conference},
Sydney, Australia,
\DatestampYM{2006}{8}.}

% Note the use of {\CharSpace} for aligning shorter numbers.
\Gap
\NumberedItem{{\CharSpace}[1]}
\href{http://www.example.com/my-paper-doi-1}
{\underline{J.~Doe} and J.~Citizen,
``Measuring the extent of climate change,''
in \textit{Proceedings of the International Climate Change Conference},
London, UK,
\DatestampYM{2005}{11}.}

%%%%%%%%%%%%%%%%%%%%%%%%%%%
%% AWARDS & SCHOLARSHIPS %%
%%%%%%%%%%%%%%%%%%%%%%%%%%%

\section
{Awards \&\newline
Scholarships}
{Awards \& Scholarships}
{PDF:AwardsAndScholarships}

\BulletItem
Dean's List,
Fall 2002 through Spring 2005,
Science College
\hfill
\DatestampY{2002} --
\DatestampY{2005}
\begin{detail}
\SubItem
For attaining a semester GPA of at least 3.75.
\end{detail}

\Gap
\BulletItem
Undergraduate Researcher Award,
Science College
\hfill
\DatestampYMD{2005}{05}{15}
\begin{detail}
\SubItem
For outstanding scientific contributions in the fields of lasers and climate change.
\end{detail}

\Gap
\BulletItem
Chess Tournament,
First Prize,
Science College
\hfill
\DatestampYMD{2003}{03}{10}
\begin{detail}
\SubItem
Awarded at the Tenth Annual Chess Tournament held during Open House.
\end{detail}

\Gap
\BulletItem
International Science Scholarship,
\hfill
\DatestampYMD{2001}{12}{10}
\newline
Global Science, Technology, Engineering, and Mathematics Foundation
\begin{detail}
\SubItem
Full-tuition scholarship with stipend for undergraduate studies.
One of 42 awardees in the world.
\end{detail}

%%%%%%%%%%%%%%%%%%%%%%%%%%%%%%%%%%%%%%%%%%%%
%% PROFESSIONAL AFFILIATIONS & ACTIVITIES %%
%%%%%%%%%%%%%%%%%%%%%%%%%%%%%%%%%%%%%%%%%%%%

\section
{Professional Affiliations\newline
\& Activities}
{Professional Affiliations \& Activities}
{PDF:ProfessionalAffiliationsActivities}

\href{http://www.example.com/my-society}
{\textbf{Society of Professional Earth Scientists}},
New York, USA

\GapNoBreak
\BulletItem
Member
\hfill
\DatestampY{2009} --
Present

%%%%%%%%%%%%%%%%%%%%%%%
%% CAMPUS ACTIVITIES %%
%%%%%%%%%%%%%%%%%%%%%%%

\section
{Campus Activities}
{Campus Activities}
{PDF:CampusActivities}

\href{http://www.example.com/my-club}
{\textbf{First Volunteers Club}},
First American University

\GapNoBreak
\BulletItem
President
\hfill
\DatestampYMD{2006}{08}{15} --
\DatestampYMD{2007}{08}{15}
\begin{detail}
\SubBulletItem
Lorem ipsum dolor sit amet, consectetur adipiscing elit.
\SubBulletItem
Curabitur vitae laoreet velit, vel ultricies est. Nam nec elit ac ante facilisis ultrices.
\SubBulletItem
Integer sit amet turpis dolor. Lorem ipsum dolor sit amet, consectetur adipiscing elit. Nunc at orci eu leo vulputate finibus sed et sem.
\SubBulletItem
Suspendisse volutpat sapien et mi cursus, gravida ornare mauris sollicitudin.
\end{detail}

%%%%%%%%%%%%%%%%%%%%%%%%%%%
%% OTHER WORK EXPERIENCE %%
%%%%%%%%%%%%%%%%%%%%%%%%%%%

\section
{Other Work\newline
Experience}
{Other Work Experience}
{PDF:OtherWorkExperience}

\href{http://www.example.com/my-company}
{\textbf{Alpha Engineering Firm}},
Oakland, Ohio, USA

\GapNoBreak
\BulletItem
Project Officer,
Research \& Development Division
\hfill
\DatestampYMD{2007}{10}{15} --
\DatestampYMD{2008}{01}{15}
\begin{detail}
\SubBulletItem
Nullam venenatis egestas nisl eget elementum.
\SubBulletItem
Nulla finibus justo vel turpis efficitur, non lacinia orci maximus. Proin rhoncus, felis vel hendrerit lacinia, enim ipsum ultricies massa, sit amet interdum nisi massa sit amet justo.
\SubBulletItem
Etiam vitae eros mollis, consectetur quam quis, molestie massa.
\end{detail}

%%%%%%%%%%%%%%%
%% LANGUAGES %%
%%%%%%%%%%%%%%%

\section
{Languages}
{Languages}
{PDF:Languages}

\BulletItem
English: Native language.

\GapNoBreak
\BulletItem
Spanish: Fluent (speaking, reading, writing).

\GapNoBreak
\BulletItem
Latin: Intermediate (reading); basic (speaking, writing).

%%%%%%%%%%%%
%% SKILLS %%
%%%%%%%%%%%%

\section
{Skills}
{Skills}
{PDF:Skills}

{\TeX}, {\LaTeX}, {\XeLaTeX},
MATLAB,
Mathematica,
R,
Adobe Photoshop,
Adobe Illustrator.

%%%%%%%%%%%%%%%
%% INTERESTS %%
%%%%%%%%%%%%%%%

\section
{Interests}
{Interests}
{PDF:Interests}

Digital photography,
typography,
swimming.

%%%%%%%%%%%%%%%%
%% REFERENCES %%
%%%%%%%%%%%%%%%%

\section
{References}
{References}
{PDF:References}

\BulletItem
\textbf{Professor Jonathan Public}
\newline
Professor of Geology and Mechanical Engineering
\newline
First American University
\newline
1000 First Avenue, Springfield, Massachusetts 22222, USA
\newline
\href{mailto:jonathanpublic@example.com}
{jonathanpublic@example.com}
\,\SubBulletSymbol\,
+1\,(555)\,222-2222

\BigGap
\BulletItem
\textbf{Dr Alice Bob Carol}
\newline
Director, Research \& Development
\newline
Alpha Engineering Firm
\newline
20 North Street, Oakland, Ohio 33333, USA
\newline
\href{mailto:alicebobcarol@example.com}
{alicebobcarol@example.com}
\,\SubBulletSymbol\,
+1\,(555)\,333-3333

%%%%%%%%%%%%%%%%%%%%%%%%%%%%%%%%%
%% SECTION WITH USAGE EXAMPLES %%
%%%%%%%%%%%%%%%%%%%%%%%%%%%%%%%%%

\section
{Section\newline
With\newline
Usage\newline
Examples}
{Section With Usage Examples (For PDF Bookmark)}
{PDF:SectionWithUsageExamples:ForPDFLink}

\subsection
{This is a Subsection}
{This is a Subsection}
{PDF:ThisIsASubSection}

\GapNoBreak
\BulletItem
Use \CodeCommand{section} and \CodeCommand{subsection} to create sections and subsections.
These will appear in the PDF bookmarks too.

\GapNoBreak
\BulletItem
This is the second \CodeCommand{BulletItem}.
Long items are automatically indented.
Lorem ipsum dolor sit amet, consectetur adipiscing elit.
Sed sed aliquam massa.
\begin{detail}
\SubBulletItem
This is a \CodeCommand{SubBulletItem}.
Long items are automatically indented.
Lorem ipsum dolor sit amet, consectetur adipiscing elit.
Sed sed aliquam massa.
Aliquam dignissim mi non enim feugiat elementum.
Donec sit amet turpis ac velit ultrices volutpat.
Aliquam vitae elit massa.
\SubBulletItem
This is the second \CodeCommand{SubBulletItem}.
\SubBulletItem
The \CodeCommand{SubBulletItem}'s are between
\CodeCommand{begin\{detail\}} and
\CodeCommand{end\{detail\}} so that they are typeset in a smaller font.
\end{detail}

\Gap
\BulletItem
This is the third \CodeCommand{BulletItem}.

\Gap
\BulletItem
A \CodeCommand{Gap} or \CodeCommand{GapNoBreak} is inserted between the \CodeCommand{BulletItem}'s so that there is a small vertical space between them.
The ``NoBreak'' version prevents page breaking, and should be used to avoid orphaned headings and other formatting issues.

\BigGap
\subsection
{This is the Second Subsection}
{This is the Second Subsection}
{PDF:ThisIsTheSecondSubSection}

\GapNoBreak
\BulletItem
A \CodeCommand{BigGap} or \CodeCommand{BigGapNoBreak} is inserted between subsections so that there is a bigger vertical space between them.
The ``NoBreak'' version prevents page breaking.

%%%%%%%%%%%%%%%%%%%%%%%%%%%%%%%%%%%%%%%%%
%% ANOTHER SECTION WITH USAGE EXAMPLES %%
%%%%%%%%%%%%%%%%%%%%%%%%%%%%%%%%%%%%%%%%%

\section
{Another\newline
Section\newline
With\newline
Usage\newline
Examples}
{Another Section With Usage Examples (For PDF Bookmark)}
{PDF:AnotherSectionWithUsageExamples:ForPDFLink}

\textbf{This is a Plain Heading},
followed by an \CodeCommand{hfill} and a date range
\hfill
\DatestampYM{2015}{10} --
\DatestampYM{2015}{12}

\GapNoBreak
\BulletItem
This is a \CodeCommand{BulletItem}.
\begin{detail}
\SubBulletItem
This is a \CodeCommand{SubBulletItem}.
\end{detail}

\GapNoBreak
\BulletItem
This is a \CodeCommand{BulletItem}.
\begin{detail}
\SubItem
This is a \CodeCommand{SubItem}, which has no bullet.
Note the alignment with the \CodeCommand{BulletItem} above.
\end{detail}

\GapNoBreak
\Item
This is an \CodeCommand{Item}, which has no bullet.
Note the alignment with the \CodeCommand{BulletItem} above.
\begin{detail}
\SubItem
This is a \CodeCommand{SubItem}.
\end{detail}

\GapNoBreak
\NumberedItem{[16]}
This is a \CodeCommand{NumberedItem}.
Note the alignment with the \CodeCommand{SubBulletItem} above.

\GapNoBreak
\NumberedItem{{\CharSpace}[6]}
This is a \CodeCommand{NumberedItem} with a \CodeCommand{CharSpace} in its argument for padding shorter numbers.
Note the alignment with the \CodeCommand{NumberedItem} above.

\BigGap
\textbf{Usage Notes}

\GapNoBreak
\BulletItem
New Lines and Paragraphs
\begin{detail}
\SubBulletItem
To create a new line within the same paragraph (i.e., with the same indentation), use \CodeCommand{newline} instead of \CodeCommand{\textbackslash}.
The latter will not work because it breaks the long table.
\SubBulletItem
To create a new paragraph, use \CodeCommand{par} or simply leave an empty line.
Paragraph indentations (from
\CodeCommand{Item},
\CodeCommand{SubItem},
\CodeCommand{BulletItem},
\CodeCommand{SubBulletItem},
etc.) do not carry across different paragraphs.
\end{detail}

\Gap
\BulletItem
Vertical Spacing Between Items
\begin{detail}
\SubBulletItem
Use \CodeCommand{Gap} or \CodeCommand{GapNoBreak} to insert a small vertical space between items within the same section.
\SubBulletItem
Use \CodeCommand{BigGap} or \CodeCommand{BigGapNoBreak} to insert a bigger vertical space between items within the same section.
\SubBulletItem
The ``NoBreak'' versions prevent page breaking.
\end{detail}

\Gap
\BulletItem
Dates
\begin{detail}
\SubBulletItem
Use
\CodeCommand{DatestampYMD\{YYYY\}\{MM\}\{DD\}},
\CodeCommand{DatestampYM\{YYYY\}\{MM\}}, and
\CodeCommand{DatestampY\{YYYY\}}
to specify dates.
\SubBulletItem
Change the date format option passed to the document class to adjust how dates are displayed throughout the document:
MMMyyyy (``Dec~2010''),
ddMMMyyyy (``31~Dec~2010''),
MMMMyyyy (``December~2010''),
ddMMMMyyyy (``31~December~2010''),
yyyyMMdd (``2010-12-31''),
yyyyMM (``2010-12''),
yyyy (``2010'').
\end{detail}

%%%%%%%%%%%%%%%%%%%%%%%%%%%%%%%%%%%
%% MULTILINGUAL UNICODE EXAMPLES %%
%%%%%%%%%%%%%%%%%%%%%%%%%%%%%%%%%%%

\section
{Multilingual Unicode Examples}
{Multilingual Unicode Examples}
{PDF:MultilingualUnicodeExamples}

\BulletItem
Assortment of unicode characters from
\href{http://www.ltg.ed.ac.uk/~richard/unicode-sample.html}
{http://www.ltg.ed.ac.uk/{\TildeSymbol}richard/unicode-sample.html}

\begin{detail}
\SubItem
\textbf{Latin Extended-A}
Ā ā Ă ă Ą ą Ć ć Ĉ ĉ Ċ ċ Č č Ď ď Đ đ Ē ē Ĕ ĕ Ė ė Ę ę Ě ě Ĝ ĝ Ğ ğ Ġ ġ Ģ ģ Ĥ ĥ Ħ ħ Ĩ ĩ Ī ī Ĭ ĭ Į į İ ı IJ ij Ĵ ĵ
\textbf{Latin Extended-B}
ƀ Ɓ Ƃ ƃ Ƅ ƅ Ɔ Ƈ ƈ Ɖ Ɗ Ƌ ƌ ƍ Ǝ Ə Ɛ Ƒ ƒ Ɠ Ɣ ƕ Ɩ Ɨ Ƙ ƙ ƚ ƛ Ɯ Ɲ ƞ Ɵ Ơ ơ Ƣ ƣ Ƥ ƥ Ʀ Ƨ ƨ Ʃ ƪ ƫ Ƭ ƭ Ʈ Ư ư Ʊ Ʋ Ƴ ƴ Ƶ
\textbf{Latin Extended Additional}
Ḁ ḁ Ḃ ḃ Ḅ ḅ Ḇ ḇ Ḉ ḉ Ḋ ḋ Ḍ ḍ Ḏ ḏ Ḑ ḑ Ḓ ḓ Ḕ ḕ Ḗ ḗ Ḙ ḙ Ḛ ḛ Ḝ ḝ Ḟ ḟ Ḡ ḡ Ḣ ḣ Ḥ ḥ Ḧ ḧ Ḩ ḩ Ḫ ḫ Ḭ ḭ Ḯ ḯ Ḱ ḱ Ḳ ḳ Ḵ ḵ
\textbf{Greek}
ʹ ͵ ͺ ; ΄ ΅ Ά · Έ Ή Ί Ό Ύ Ώ ΐ Α Β Γ Δ Ε Ζ Η Θ Ι Κ Λ Μ Ν Ξ Ο Π Ρ Σ Τ Υ Φ Χ Ψ Ω Ϊ Ϋ ά έ ή ί ΰ α β γ δ ε ζ η θ
\textbf{Cyrillic}
Ё Ђ Ѓ Є Ѕ І Ї Ј Љ Њ Ћ Ќ Ў Џ А Б В Г Д Е Ж З И Й К Л М Н О П Р С Т У Ф Х Ц Ч Ш Щ Ъ Ы Ь Э Ю Я а б в г д е ж з
\textbf{Hebrew}
א ב ג ד ה ו ז ח ט י ך כ ל ם מ ן נ ס ע ף פ ץ צ ק ר ש ת װ ױ ײ ֝ ֞ ֟ ֠ ֡ ֣ ֤ ֥ ֦ ֧ ֨ ֩ ֪ ֫ ֬ ֭ ֮ ֯ ְ ֱ ֒ ֓ ֔
\textbf{Armenian}
{\UseSecondaryFont
Ա Բ Գ Դ Ե Զ Է Ը Թ Ժ Ի Լ Խ Ծ Կ Հ Ձ Ղ Ճ Մ Յ Ն Շ Ո Չ Պ Ջ Ռ Ս Վ Տ Ր Ց Ւ Փ Ք Օ Ֆ ՙ ՚ ՛ ՜ ՝ ՞ ՟ ա բ գ դ ե զ}
\textbf{Thai}
{\UseSecondaryFont
ก ข ฃ ค ฅ ฆ ง จ ฉ ช ซ ฌ ญ ฎ ฏ ฐ ฑ ฒ ณ ด ต ถ ท ธ น บ ป ผ ฝ พ ฟ ภ ม ย ร ฤ ล ฦ ว ศ ษ ส ห ฬ อ ฮ ฯ ะ ั า ำ ิ}
\end{detail}

\end{body}

%%%%%%%%%%%
% CV NOTE %
%%%%%%%%%%%

\UseNoteFont%
\null\hfill%
[\textit{\CVNote}]%
\hspace{2.0mm}\null

\end{document}
